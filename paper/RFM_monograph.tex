\documentclass[11pt]{article}

\usepackage[margin=1in]{geometry}
\usepackage{microtype}
\usepackage{graphicx}
\usepackage{subcaption}
\usepackage{booktabs}
\usepackage{amsmath, amssymb, amsthm}
\usepackage{textcomp}
\usepackage{algorithm}
\usepackage{algorithmic}
\usepackage{enumitem}
\usepackage{mathtools}
\usepackage{xcolor}
\usepackage{hyperref}
\usepackage[capitalize,noabbrev]{cleveref}

\hypersetup{
    colorlinks = true,
    linkcolor  = blue,
    citecolor  = teal,
    urlcolor   = magenta
}

\setlist{nosep,leftmargin=1.5em}

\theoremstyle{plain}
\newtheorem{theorem}{Theorem}[section]
\newtheorem{lemma}[theorem]{Lemma}
\newtheorem{proposition}[theorem]{Proposition}
\newtheorem{corollary}[theorem]{Corollary}

\theoremstyle{definition}
\newtheorem{definition}[theorem]{Definition}
\newtheorem{assumption}[theorem]{Assumption}

\theoremstyle{remark}
\newtheorem{remark}[theorem]{Remark}

\title{
Routing Foundation Model (RFM):\\
A Unified Neural Optimization Framework\\
for Large-Scale Routing and MILPs
}

\author{
  Ritwika Kancharla\\
  Independent Researcher\\
  \texttt{ritwikareddykancharla@gmail.com}
}

\date{\today}

\begin{document}

\maketitle

\begin{abstract}
Large-scale routing and supply-chain systems such as Amazon's
middle-mile network are routinely modeled as mixed-integer linear
programs (MILPs) with many binary variables and tight operational
constraints. Classical solvers provide high-quality solutions but
are often too slow for real-time re-optimization, while existing
Neural Combinatorial Optimization models do not explicitly encode
MILP structure or constraint geometry. This monograph proposes the
\emph{Routing Foundation Model (RFM)}, a unified neural optimization
framework that treats transformer-style architectures as learned
surrogate solvers for routing MILPs. RFM combines MILP-aware
encoders, dual-informed attention via violation signals $Ax - b$,
constraint-specialized Mixture-of-Experts, diffusion-style priors
over routing structures, and world-model components for multi-step
logistics planning. We outline the framework, connect it to
primal--dual and proximal optimization, and discuss experimental
protocols and open problems toward foundation models for routing.
\end{abstract}

\tableofcontents
\newpage

\section{Introduction}

% TODO: paste / adapt your existing intro + motivation text here.

\section{Background: Routing, MILPs, and Neural Optimization}

\section{Transformers Through the Lens of Optimization}

\section{Problem Formulation: Routing MILPs at Scale}

\section{Routing Foundation Model (RFM): High-Level Architecture}

\section{Component I: MILP-Transformer as Surrogate Solver}

\section{Component II: Neural Routing Optimization Model (NROM)}

\section{Component III: Diffusion Priors for Routing}

\section{Component IV: Routing World Models}

\section{Training, Evaluation, and Benchmarks}

\section{Relation to Prior Work}

\section{Discussion and Open Problems}

\section{Conclusion}

\bibliographystyle{plainnat}
\bibliography{references}

\appendix

\section{Additional Derivations}

\section{Implementation Details}

\end{document}
